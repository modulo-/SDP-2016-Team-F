\section{Low-Level Commands}

The arduino internally maintains a queue of low-level commands which capture
what the robot is currently doing. Once a command finishes, the arduino
immediately moves on to the next. Structurally, low-level commands work
similarly to high-level ones. They start with an opcode, depending on which
some arguments may follow. The available low-level commands are:

\begin{tabularx}{\textwidth}{rlX}
    Opcode & Arguments & Description \\
    \texttt{0x00} & time (\verb$uint16_t$) &
        Waits for the given time \\
    \texttt{0x01} & time (\verb$uint16_t$) &
        Brakes for the given time \\
    \texttt{0x02} & time (\verb$uint16_t$) &
        Opens the grabbers for the given time \\
    \texttt{0x03} & time (\verb$uint16_t$) &
        Closes the grabbers slowly for the given time \\
    \texttt{0x04} & time (\verb$uint16_t$) &
        Closes the grabbers quickly for the given time \\
    \texttt{0x05} & time (\verb$uint16_t$) &
        Activates the kicker for the given time \\
    \texttt{0x08} & distance (\verb$int16_t$) &
        Moves right by the given distance \\
    \texttt{0x09} & angle (\verb$int16_t$) &
        Spins clockwise by the given angle \\
    \texttt{0x0b} & speed (\verb$uint8_t$) &
        Sets the global speed modifier to the given value \\
%    \texttt{0x0c} & distance (\verb$int16_t$), front wheel weight (\verb$uint16_t$), back wheel weight (\verb$uint16_t$) &
%        Moves the given distance, with the speed of the front and back wheels being in the specified proportion (used for arc movement) \\
    \texttt{0x7f} & \textit{none} & Does nothing \\
\end{tabularx}

Further, the most significant bit of a low-level commands opcode signifies if this command is interruptable or not. If it is 1, the command is not interruptable.

The arduino periodically updates the instruction list to decrease the time
argument of the current instruction by the time passed, or, if the instruction
is finished, moving onto the next one. If a new command sequence arrives, it is
inserted in place of the first interruptable command in the sequence, with
this, and all later commands being discarded. This ensures that
non-interruptable commands are always run, and are run even if they consist of
complex sequences which should not be interrupted. This is also the main reason
for a no-op existing; it allows seperating multiple sequences of
uninterruptable commands.

Instructions setting values relating to the comms test are set immediately,
and the arduino moves on to the next instruction. The send instruction
\texttt{0xf2} suspends normal operation and executes a writing the buffer to
the I$^2$C port with the previously set delay.
