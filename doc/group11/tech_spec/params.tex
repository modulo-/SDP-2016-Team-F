\section{Command-Line Parameters} \label{params}

The following command-line parameters are supported, and while none are
required the \texttt{--defender}, \texttt{--attacker}, \texttt{--color},
\texttt{--plan}, \texttt{--goal}, and \texttt{--pitch} options should all be
set for normal operation.

\vspace{5mm}

\begin{tabularx}{\textwidth}{ll|X}
    \textbf{Short} & \textbf{Long} & \textbf{Description} \\
    \hline
    \texttt{-h} & \texttt{--help}     & Shows a brief help text. \\
    \texttt{-z} & \texttt{--visible}  & Displays logging in purple. \\
    \texttt{-1} & \texttt{--defender} & Takes one argument, for the path of the
    defender robots USB RF device. \\
    \texttt{-2} & \texttt{--attacker} & Takes one argument, for the path of the
    attacker robots USB RF device. \\
    \texttt{-l} & \texttt{--logging}  & Takes a comma separated list of
    logging levels to use as an argument. Supported logging levels are `debug',
    `info', `warning', and `error'. \\
    \texttt{-c} & \texttt{--color}    & Takes the color of our team as the
    argument, either `blue', or `yellow' (or `b' and `y' for short) should be
    used. \\
    \texttt{-p} & \texttt{--plan}     & Takes the plan to use as an argument.
    Can be used to run subroutines other than playing the game, however for
    normal operation `game' is the correct value to use. \\
    \texttt{-g} & \texttt{--goal}     & Takes the side our goal is on as an
    argument. Use either `left' or `right'. \\
                & \texttt{--pitch}  & Takes the pitch number being used as an
    argument, in order to enable pitch specific calibrations. Use `0' for room
    3.D03, and `1' for room 3.D04. \\
\end{tabularx}
