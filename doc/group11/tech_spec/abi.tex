\newcommand{\abispec}[6]{
\subsection{\texttt{#1}}

\begin{tabularx}{\textwidth}{>{\hsize=.5\hsize}XX}
\textbf{Instruction byte:}               & \texttt{#2} \\
\textbf{Argument bytes:}                 & #3 \\
\textbf{Argument type(s):}               & \texttt{#4} \\
\textbf{Action:}                         & #5 \\
\textbf{Relation to low-level commands:} & #6 \\
\end{tabularx}
}

\section{List of ABI Commands} \label{abi}

This section will outline the available instructions for the communications
ABI, the arguments they take, and the effect they will produce. A note on the
notation for arguments: The arguments are standard C types, and are read from
the ABI buffer as this type directly. Since the arduino is little-endian, this
for example means that \texttt{0x03 29}, read as a \texttt{uint16\_t}, is
10499.

\abispec{WAIT}{0x00}{2}{uint16\_t}{Waits for the given time in milliseconds}
    {Maps one-to-one to the low-level WAIT command}
\abispec{BRAKE}{0x01}{2}{uint16\_t}
    {Brakes all motors for the given time in milliseconds}
    {Relation to low-level commands: Maps one-to-one to the low-level
     \texttt{BRAKE} command}
\abispec{STRAIT}{0x02}{2}{int16\_t}
    {Moves right the given distance in millimeters. This distance may be
     negative, in which case the robot moves left instead}
    {Relation to low-level commands: Maps to an equivalent low-level
     \texttt{STRAIT} command, followed by a low-level \texttt{BRAKE} command}
\abispec{SPIN}{0x03}{2}{int16\_t}
    {Rotates clockwise on the spot for the given number of minutes (60 minutes
     corresponds to one degree). This may be negative, in which case the robot
     rotates counter-clockwise instead.}
    {Maps to an equivalent low-level \texttt{SPIN} command, followed by a
     low-level \texttt{BRAKE} command}
\abispec{KICK}{0x04}{2}{uint16\_t}
    {Opens the grabbers and actuates the kicker for the given time in
     milliseconds}
    {First, a \texttt{GRABBER\_FORCE} is issued to ensure the ball is in front
     of a kicker. Then, a part-way \texttt{GRABBER\_OPEN} is done, followed by
     the equivalent low-level \texttt{KICK}. Finally, a \texttt{GRABBER\_OPEN}
     is issued to finish opening the grabbers. All of these are
     uninterruptable.}
\abispec{GRABBER\_OPEN}{0x06}{0}{\textnormal{\textit{none}}}{Opens the grabbers}
    {Maps one-to-one to an uninterruptable low-level \texttt{GRABBER\_OPEN}}
\abispec{GRABBER\_CLOSE}{0x07}{0}{\textnormal{\textit{none}}}
    {Closes the grabbers in multiple stages to ensure the ball is caught well}
    {First, a \texttt{GRABBER\_CLOSE} is issued to get the ball close to the
     solonoid.  Then, a \texttt{GRABBER\_OPEN}, followed by a
     \texttt{GRABBER\_FORCE} is issued, to ensure that the ball presses the
     solonoid in. Finally, since the \texttt{GRABBER\_FORCE}'s power leads to
     the grabbers bouncing back to being partially open, a
     \texttt{GRABBER\_CLOSE} is issued.}
