\newcommand\visiontodo[1]{\textbf{TODO: #1}}
% Disable todos
%\renewcommand\visiontodo[1]{}

\section{Vision}

\visiontodo{Add new section or subsection covering predictor system}

The vision system was designed to be simple and fast, both to aid in development and to minimise the delay of a movement on the
pitch being registered by the system. Originally a system was being developed from the ground up, but after a number of failures
and difficulties the decision was made to use Group 12's working system. Group 12's system was adapted from a previous year's
system and is the system used for the rest of the project.

\visiontodo{Find out which code Group 12's system was based on}

\subsection{Theory}

\visiontodo{Probably needs revision, is a bit clumsy in places}

The task for the vision system is simple: to find and report of the position of the robots (or more accurately, their coloured
top plates) and the ball from an overhead, colour image of the pitch. The design of our system addresses this as follows. First
coloured areas matching the colours of the ball and top plate markers are identified in the image. These areas are then checked
for certain properties to determine if they are the ball, part of a top plate or random noise. For the ball these properties
are size related, and for the top plate these properties are related to the proximity of other markers of expected colours.
When an area of the image is identified as an area of interest this area is reported. When a certain object is not found the
system does not report anything and this emission is handled by the prediction system.

\subsection{Implementation}

\visiontodo{High-level overview of implementation, discussion of legacy code}

\visiontodo{High-level diagram showing flow of information through vision system}

\visiontodo{Appendix: screenshots for various stages of processing?}

\subsection{Problems}

\visiontodo{A brief discussion of the systems shortcomings and possible areas for improvement?}
