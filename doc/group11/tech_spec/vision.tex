\newcommand\visiontodo[1]{\textbf{TODO: #1}}
% Disable todos
%\renewcommand\visiontodo[1]{}

\section{Vision}

The vision system was designed to be simple and fast, both to aid in development and to minimise the delay of a movement on the
pitch being registered by the system. Originally a system was being developed from the ground up, but after a number of failures
and difficulties the decision was made to use Group 12's working system. Group 12's system was adapted from a previous year's
system and is the system used for the rest of the project.

\visiontodo{Find out which code Group 12's system was based on}

\subsection{Theory}

The task for the vision system is simple: to find and report of the position of the robots (or more accurately, their coloured
top plates) and the ball from an overhead, colour image of the pitch. The design of our system addresses this as follows. First
coloured areas matching the colours of the ball and top plate markers are identified in the image. These areas are then processed
to determine if they are the ball, part of a top plate or random noise. For the ball the size of the area is the deciding factor,
and for top plate markers the proximity of other areas of colours that would be expected for a top plate are factored in. When
an area of the image is identified as an area of interest this area is reported. When a certain object is not found the system
does not report anything and this emission is handled by the prediction system.

\visiontodo{Is predictor system covered by planning section?}

\subsection{Implementation}

The vision system is implemented as a Python module largely separate from the rest of the system. The vision system is run in
a separate thread from the planning system and communicates with it through a shared world state object. The OpenCV library
was used to handle the overhead camera feed and perform the majority of the vision related operations.

The vision system follows a basic flow of information: the raw camera frame is retrieved, preprocessed, information is extracted,
this information is passed to the planning system, repeat. After retrieval radial distortion is removed from the raw image. The image
is then blurred using a Gaussian blur to reduce noise in the image. This undistorted and reduced noise image is then converted to the
HSV (Hue Saturation Value) colour space before being passed to information retrieval. The HSV colour space is used instead of the default
BGR (Blue Green Red) as the Hue component represents the colour largely independent of brightness and intensity, allowing a particular colour
(or colour range) to be identified more accurate than could be using BGR.

From the HSV image, a binary image is produced for each of the colour ranges used to identify objects (a red colour range for the ball, a pink one
for the pink top plate markers, etc.). This binary image is the same size as the HSV image and contains true (or equivalent value) when the pixels
in the HSV value are in the colour range and false (or equivalent value) elsewhere in the image. Contours are then extracted from these binary images
and these contours are then preprocessed further to determine if they are a ball or top plate marker. Contours are vector representations of the edges in
an image, so in this case describe the shapes of the true areas in the binary images. To determine is a red contour is the ball, the contour is first
checked to ensure it is not too small according to a configurable value. The enclosing rectangle that has minimum area is then calculated. If this rectangle
has close to equal length side (within a certain threshold), the contour is assumed to represent the ball. To find the top plate markers, the same process is
followed. The found markers are then grouped with those within close proximity to the centre marker. The orientation of the top plate is calculate by calculating
the angle between the centre marker and the 'odd one out' (the outer marker of a different colour) which is then rotated 45° to give the orientation of the
front of the top plate.

After this information has been extracted from the binary images, it is placed in the world state object to be passed to the planning system. The vision system
then retrieves the next frame and repeats the process.

\visiontodo{High-level diagram showing flow of information through vision system}

\visiontodo{Appendix: screenshots for various stages of processing?}

\subsection{Problems}

\visiontodo{A brief discussion of the systems shortcomings and possible areas for improvement?}
