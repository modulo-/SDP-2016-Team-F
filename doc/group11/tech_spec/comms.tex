\section{Communications} \label{comms}

\subsection{Data Transfer}

Communications between the arduino and computer are done over the supplied RF
link. The frequency is set to the group frequency specified. Further, the
optional encryption was enabled, and the PAN ID set to `6810'.

Data is sent to and from the devices in packets, where the basic form of a
packet is `\verb$<target><source><data><checksum>\r\n$'. \texttt{<target>} and
\texttt{<source>} are the ASCII character `1' for the our robot, and the ASCII
character `c' for the computer. For robot to computer messages, the targets `d'
and `e' are also valid, used to send debug and error messages respectively.
\texttt{<data>} is the base64 encoded binary message, and \texttt{<checksum>} a
checksum of the preceding message. The base64 encoding does not use padding
bytes, but is otherwise standard.

The checksum consists of 2 characters, which are calculated as the checksum of
all characters at even (starting 0) and odd indicies respectively, including
\texttt{<target>} and \texttt{<source>}. These checksums are done by retrieving
the value from 0-63 of the character in base64 encoding, XORing these together.
The resulting value from 0-63 is then encoded back into a base64 character.

For example, the checksum of `\texttt{dt6bas2}' is `\texttt{La}'.

Devices ignores all packets not addressed to it, as well as malformed packets
(including those with an incorrect checksum). Upon receipt of a packet, the
device sends an acknowledgement. An acknowledgement packet has the form
`\verb%<target>$<checksum>\r\n%'. An acknowledgement packet is not itself
acknowledged.

The computer side will resend packets which have not recieved an
acknowledgement. The arduino side does not, and discards packets which are
longer than the internal buffer (60 bytes). To prevent flooding, only the
previously sent packet is resent. This suits the application well, as
previously sent packets contain intructions to the robot which are supersceded
by the more recent one.

\subsection{Instruction ABI}

For communications, the arduino and the controlling computer share a binary
interface, with the computer sending binary packets of instructions to the
arduino, and the arduino executing them. Packets sent by the computer consist
of one or more instructions for the arduino, supplied in sequence. The first
byte of the tranmitted data marks the start of the first instruction. Then, the
byte following the end of the current instruction marks the start of the next.
Instructions are variable in length, but always begin with one byte identifying
the type of the instruction. This is then followed by zero or more bytes of
arguments. For example, the binary data \texttt{0x06 02 ff 00} consists of two
instructions: \texttt{0x06} and \texttt{0x02 ff 00}. \texttt{0x06} is the
instruction to open the grabbers, and takes no further arguments. \texttt{0x02}
is the instruction to move left or right, and takes two more bytes as its
argument. \texttt{0x02 ff 00 06} contains the same instructions, but in a
different order. \texttt{0x06 ff 02 00} however has undefined behaviour, as
there is no instruction \texttt{0xff}. The instructions of this ABI are
documented in \cref{abi}.
