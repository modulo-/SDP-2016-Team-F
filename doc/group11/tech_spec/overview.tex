\section{Architectural Overview}

The core control of the robot was split across an external PC and the onboard
Arduino Xino RF unit. An analog video feed with a top-down perspective of the
pitch was supplied to the external PC. The external PC is able to communicate
via a supplied radio frequency link. The PC uses a USB RF device, while the
Arduino has an inbuilt RF transmitter and receiver.

The external PC is responsible for all decision making, while the Arduino
carries out instructions passed to it as quickly and accurately as possible.
To begin with, the PC listens for new frames from the video input, and
processes these to extract a set of coordinates for objects of interest: The
robots (which are detected by their coloured top-plates), and the ball. This is
used to update in internal model of the world. A few adjustments are made to
smooth over misdetections by the vision system and correct positions to account
for the fact that the robot top-plates are elevated.

At set intervals, the planning module runs, and takes a snapshot of the current
world state. It then determines what course of action to take in the current
situation, and what concrete instructions should be sent to the robot. These
instructions are then encoded into the instruction ABI, and transmitted via the
RF link (see \cref{comms} for more details). Finally, the Arduino receives the
instructions and sets the outputs to the motors and the kickers accordingly.
Some amount of feedback from the rotary encoders on the motors allows the
Arduino to adjust the motor powers on the fly, to correct its movement. The
Arduino determines itself when an instruction is over, and proceeds with the
next one, or, if no instructions are left, remains idle.
