\section{Planning}

\subsection{Design}

The planner makes its decision based on the world state passed to it. By
default, this is done once per second. However, the interval between planner
decisions may be changed based on the last executed action. Some actions, such
as turning, require less time than one second. Other actions, however, such as
moving and closing the grabbers require significantly more time. The planner
estimates this time, and sets the delay to it.

The planner uses a system of goals and actions, the hierarchy of which may be
seen in Figure \ref{fig:goalsandactionsstructure}. Goals define an overall
strategy in a given situation, leading to a particular aim, and are composed
of a sequence of actions. In each step, the planner selects an appropriate
action to achieve its aim by traversing a list of actions in the currently
selected goal. This selection is done by evaluating preconditions of each
action. In general, goals are high-level and do not change as often as actions.
%A list of goals may be found in \cref{goalstructure}.

Actions define usually one instruction given to the robot with preconditions
for its execution. As already mentioned, an action can also return a specific
delay time to postpone or speed up next planner execution. Actions compute any
parameters to the intructions they send themselves, e.g. there may be an action
to rotate to a point, but not one to rotate 30\degree.

\begin{figure}[H]
	\begin{center}
    \includegraphics[width=1.0\linewidth]{res/goalandactions.png}
    \caption{The overall planner structure. The planner consists of multiple goals. Each goal has a list of actions which contain two elements - preconditions and a method to execute.}
    \label{fig:goalsandactionsstructure}
	\end{center}
\end{figure}

\subsection{Goal selection}

The process of selecting a new goal follows a procedure described in Figure
\ref{fig:goalpicker}. In every step of the planner, the current world state is
processed and resulting goal is generated based on this analyses. There are
three main conditions that effect this decision.


\begin{figure}[H]
	\begin{center}
    \includegraphics[width=0.60\linewidth]{res/goalpicker.png}
    \caption{A flowchart showing all main conditions effecting planner's decision to select a goal in each step.}
    \label{fig:goalpicker}
	\end{center}
\end{figure}

\pagebreak

First of all, if our robot captures the ball, it will try to pass the ball to
our teammate. If the ball is not in our robot's possession, the planner
has to strategically decide where to go based on the ball's position and the
position of all other robots. If our robot could approach the ball faster than
all other robots (our attacker included), then it would attempt to grab the
ball. Otherwise, it would switch to the \texttt{Defend} goal and start
defending our goal area. This involves returning to our defence zone. If
our robot decides to grab the ball and the situation changes so that any
other robot is closer than ours, the planner immediately changes the goal to
\texttt{Defend}. Also, if the ball is heading towards our goal and there is a
high chance of scoring, the planner would choose to defend over trying to catch
the ball. Finally, one special condition was added. If the ball is too close to
our goal, Tractor Crab's goal changes to \texttt{Idle}. The reasoning behind
this is that it is very unlikely that the robot will grab the ball safely, and
it is more likely for an attempt to result in an own-goal.

\subsection{Goals}

\subsubsection{\texttt{GrabBall}}

\texttt{GrabBall} is one of the principal goals in the planner. Its purpose
is to move to the best position for grabbing the ball and eventually grab the
ball. Since the robot can only move sideways, the task is not trivial, and
requires a sideways approach detailed in Figure \ref{fig:sidewaysgrabball}.

The naive way of grabbing the ball would be to approach the ball, spin it so
that the robot faces the ball, and grab it. However, this approach faces a
substantial shortage of hitting the ball while spinning close-by. Thus, it
needs to rotate first and then approach the ball. First, the planner finds
points suitable for grabbing the ball, assuming that a robot placed on a point
is facing the ball. There are usually two appropriate points -- catch-points. 


One of the advantages of sideways movement is that neither side has priority
and thus both sides can be used for any kind of task. As long as the robot
faces the ball, it can move in both directions. Thus, there are 4 possible
rotations - one for each catch-point and for each movement direction.  Only two
of these rotations lead to a situation when the robot faces the ball. Out of
these, the planner picks the one that requires the smaller angle of rotation.
Once the planner knows the catch-point and the corresponding rotation, it can
approach the ball. This process is shown in Figure
\ref{fig:sidewaysgrabballangle}. Figure \ref{fig:sidewaysgrabballdone} shows
the situation when the robot is in front of the ball and is ready to grab it.


\begin{figure}[H]
	\begin{center}
	\begin{subfigure}{0.5\textwidth}
  		\includegraphics[width=1.0\textwidth]{res/sidewaysgrabballangle.png}
  		\caption{The planner finds two catch-points P1 and P2. For each point, there is a corresponding angle $\phi$, which represents the rotation for approaching a point.}
  		\label{fig:sidewaysgrabballangle}
	\end{subfigure}%
	\begin{subfigure}{0.5\textwidth}
  		\includegraphics[width=1.0\textwidth]{res/sidewaysgrabballdone.png}
  		\caption{The catch-point P2 was selected, the robot rotated and moved so that it stands on P2 now. The next action would be to grab the ball. }
  		\label{fig:sidewaysgrabballdone}
	\end{subfigure}%
	
	\caption{Two charts representing the ball-catching algorithm. }
	\label{fig:sidewaysgrabball}
	\end{center}
\end{figure}

Two obstacles had to be overcame in development. Those were mostly caused by
vision inaccuracies and a dynamic environment. Firstly, our robot can overshoot
in movement and miss the catch point. If this happens, it does not recompute a
new catch-point. Instead, it tries to move back towards the current
catch-point. Secondly, our robot can overshot in rotation. Since the rotation
angle is crucial for successful movement towards a catch-point, the only
possible solution is to rotation again. However, this can lead to situations
when the robot keeps oscillating between two angles. Since such behaviour is
undesirable, a dynamic error threshold is used. As long as the robot is within
an accepted error, the rotation is accepted, with the threshold value increases
after every successive unsuccessful rotation. This approach represents a
trade-off between accuracy and speed. It is better to make a decision fast,
even if it is not optimal, because the environment constantly changes. A found
solution can be outdated if the decision/adjusting process takes too long.

\subsubsection{\texttt{Defend}}

The primary role of \texttt{Defend} goal is to prevent the opposing team from
successfully moving the ball over the defended goal-line. This is accomplished
by the robot's moving into the path of the ball and either catching it or
directing it away from the goal line. To improve chances of catching the ball
during the \texttt{Defend} goal, our robot is always trying to predict the
ball's trajectory and stand at the most suitable defending spot. This spot,
also called a defending point, is calculated as an intersection of a predicted
ball's trajectory and a semi-circle surrounding the goal area. Such point is
shown in Figure \ref{fig:defendingpoint}.

\begin{figure}[H]
	\begin{center}
    \includegraphics[width=0.5\linewidth]{res/defending.png}
    \caption{A football pitch with a robot using the \texttt{Defend} goal. The blue line represents the defending semi-circle. The red line shows an expected ball trajectory. Point D represents the defence point: the intersection of these lines.}
    \label{fig:defendingpoint}
	\end{center}
\end{figure}

Because of vision inconsistency on different parts of a pitch, the defending
point is sometimes slightly off. This problem is dealt with by oscillating
movement around the defending point. Since the robot is always moving from side
to side, it covers more area and thus its chances to catch the moving ball are
higher. A minor improvement was implemented for situations when the robot is not
close to the defending point and the ball is moving towards our goal. In this
case, the robot immediately moves to the side of the ball and tries to direct
it away.
