\section{Planning}

\subsection{Design}

The planner uses a system of goals and actions. Goals define an overall strategy in a given situation leading to a particular aim. Actions define one instruction given to the robot with preconditions for its execution. Goals are composed of a sequence of actions. A goal object selects the next action required to achieve its aim by traversing its ordered list of actions and evaluating their preconditions.

The planner makes its decision based only on the world state passed to it. This is described by an instance of the \emph{World} class passed to the planner. This object's state is updated by passing a new set of positions (as a dictionary of vectors) to the \texttt{update\_positions} method of \emph{World}. The \emph{World} class describes the state of the world from the vision, including position vectors for the robots, the ball and the goals. The \emph{World} class and its associated classes also provide methods on their data providing the planner with information about the world. Further utility functionality can be found in \emph{utils.py}.

The overall planner is used by calling \texttt{plan\_and\_act(world)}. This selects a goal based on the given world state using its \texttt{get\_goal()} method. From this goal an action is generated using \texttt{generate\_action()}. The method then runs (using \texttt{actuate(action)} and returns a delay giving the time until the planner should be run again.


\pagebreak
\subsection{Goal selection}

The process of selecting a new goal follows a procedure described in Figure \ref{fig:goalpicker}. In every step of the planner, the current world state is processed and resulting goal is generated based on this analyses. There are three main conditions that effect this decision.


\begin{figure}[H]
	\begin{center}
    \includegraphics[width=0.60\linewidth]{res/goalpicker.png}
    \caption{A flowchart showing all main conditions effecting planner's decision to select a goal in each step.}
    \label{fig:goalpicker}
	\end{center}
\end{figure}

\pagebreak

First of all, if our robot captures the ball, it will try to pass the ball to a teammate robot. If the ball is not in our robot's possession, then the planner has to strategically decide where to go based on the ball's position and the position of all other robots. If our robot could approach the ball faster than all other robots (our attacker included), then it would attempt to grab the ball. Otherwise, it would switch to \emph{Defend} goal and start defending our goal area. This involves returning to our defensive zone. If Tractor Crab decides to grab the ball and the situation changes so that any other robot is closer than ours, the planner immediately changes the goal to defend. Also, if the ball is heading towards our goal and there is a high chance of scoring, the planner would choose to defend over trying to catch the ball. Finally, one special condition was added. If the ball is too close to our goal, Tractor Crab's goal changes to \emph{Idle}. The reasoning behind this is that it is very unlikely that the robot will grab the ball safely. In other words, it is very likely that we will score on ourselves, and thus, it is preferable to do nothing.









\pagebreak
\subsection{Goals}

Figure \ref{fig:receiveandpass} shows an example of Defender's goal \emph{ReceivingPass}. It shows a list of actions that are in a queue starting from the top. When action's preconditions are met, the action is executed and the process finishes. If any of the precondition is false, the next action in the queue is processed.

\begin{center}

\begin{figure}[H]
\centering
\begin{tabular}{ | l | p{110mm} | }
\hline
\textbf{Action} & \textbf{Preconditions} \\ \hline
GrabBall &
\begin{compactitem}
\item Defender can catch ball
\item Defender's grabbers are open
\end{compactitem}  \\ \hline

GoToStaticBall &
\begin{compactitem}
\item Defender is facing the catch point
\item Defender's grabbers are open
\item Ball is static
\end{compactitem}  \\ \hline

TurnToCatchPoint &
\begin{compactitem}
\item Defender's grabbers are open
\end{compactitem}  \\ \hline

WaitForBallToCome &
\begin{compactitem}
\item Defender is facing ball
\item The ball can reach the robot
\item Defender can reach the ball
\item Grabbers are open
\item The ball is moving
\end{compactitem}  \\ \hline

FollowBall &
\begin{compactitem}
\item Defender's grabbers are open
\item The ball is moving
\end{compactitem}  \\ \hline

OpenGrabbers &
\begin{compactitem}
\item Defender's grabbers are closed
\end{compactitem}  \\ \hline

FaceFriendly &
\begin{compactitem}
\item Friendly is on the pitch
\end{compactitem}  \\ \hline
\end{tabular}
\par
\bigskip

\caption{ReceivingAndPass goal in the defender's planner. Shows actions and their corresponding preconditions that have to be satisfied.}

	\label{fig:receiveandpass}
\end{figure}
\end{center}

\pagebreak
\subsubsection{Grab Ball}

\emph{Grab ball} is one of the principal goals in the planner. Its main purpose is to move to the best position for grabbing the ball and eventually grab the ball. Since the robot can only move sideways, the task is not trivial. Figure \ref{fig:sidewaysgrabball} depicts how this was done.

\begin{figure}[H]
	\begin{center}
	\begin{subfigure}{0.5\textwidth}
  		\includegraphics[width=1.0\textwidth]{res/sidewaysgrabballangle.png}
  		\caption{Planner finds two catch-points P1 and P2. For each point, there is a corresponding angle $\phi$, which represents the rotation for approaching a point.}
  		\label{fig:sidewaysgrabballangle}
	\end{subfigure}%
	\begin{subfigure}{0.5\textwidth}
  		\includegraphics[width=1.0\textwidth]{res/sidewaysgrabballdone.png}
  		\caption{Catch-point P2 was selected, the robot rotated and moved so that it stands on P2 now. The next action would be to grab the ball. }
  		\label{fig:sidewaysgrabballdone}
	\end{subfigure}%
	
	\caption{Two charts representing the ball-catching algorithm. }
	\label{fig:sidewaysgrabball}
	\end{center}
\end{figure}


The naive way of grabbing the ball would be to approach the ball, spin it so that the robot faces the ball, and grab it. However, this approach faces a substantial shortage of hitting the ball while spinning close-by. Thus, it needs to rotate first and then approach the ball. First, the planner finds points suitable for grabbing the ball, assuming that a robot placed on a point is facing the ball. There are usually two appropriate points - \emph{catch-points}. 

One of the advantages of sideways movement is that neither side has priority and thus both sides can be used for any kind of task. As long as the robot faces the ball, it can move in both directions. Thus, there are 4 possible rotations - one for each catch-point and for each movement direction. Certainly, only two of these rotations lead to a situation when the robot faces the ball. Out of these, the planner picks the one that requires the smaller angle of rotation. Once the planner knows the catch-point and the corresponding rotation, it can approach the ball. This process is shown in Figure \ref{fig:sidewaysgrabballangle}. Figure \ref{fig:sidewaysgrabballdone} shows the situation when the robot is in front of the ball and is ready to grab it.

Two obstacles had to be overcame in development. Those were mostly caused by vision inaccuracies and dynamic environment. Firstly, Tractor Crab can overshoot in movement and miss the catch point. If this happens, it does not recompute a new catch-point. Instead, it tries to move back towards the current catch-point. Secondly, Tractor Crab can overshot in rotation. Since rotation angle is crucial for successful movement towards a catch-point, the only possible solution is to repeat the rotation again. However, this can lead to situations when the robot keeps rotating there and back. Since such behavior is undesirable, a \emph{dynamic threshold} is used. As long as the robot is within an accepted error, the rotation is accepted. Also, the threshold value increases after every successive unsuccessful rotation. This approach represents a trade-off between accuracy and speed. It is better to make a decision fast, even if it is not optimal, because the environment constantly changes. A found solution can be outdated if the decision/adjusting process takes too long.



\pagebreak
\subsubsection{Defend}



\subsection{Extensibility}

Extending the planner is simply a case of adding goals and actions then adding the logic to select these in \texttt{select\_goal(world)}. Any new goals or actions should subclass Goal and Action respectively and override methods were stated.

In actions, the logic for performing should be placed in \texttt{perform(comms)}. This method is passed a \texttt{CommsManager} object through which the robot can be sent instructions. No more than one call should be made to this object in any given action. A new action should also override \texttt{get\_delay} giving an appropriate delay (in seconds) before the planner should run again. New actions can also have preconditions defined in a variable \texttt{preconditions}.

In goals, it may only be neccessary to write a Goal subclass with an ordered list of actions (from last to first). Otherwise the \texttt{generate\_action()} method can be overriden but similar logic should be followed.

\subsection{Integration}

An instance of the \texttt{Planner} is kept by the ``main.py'' script. Based on the delays given by the planner, it calls the planner at varying intervals, passing it the latest world model provided by the vision. The planner uses a \texttt{CommsManager} object to make control the robot.
