\documentclass[a4paper]{scrartcl}
\author{Thomas Kerber, Marek Strelec}
\title{Technical Specification}
\subtitle{Group 11}

\usepackage{tabularx}
\usepackage{enumitem}
\usepackage{paralist,array}
\usepackage{listings}
\usepackage{float}
\usepackage{cleveref}
\lstset{language=Python, showstringspaces=false}

\begin{document}
\maketitle


Communications between the Arduino and computer are done over the supplied RF link. The frequency is set to the group frequency specified. 
Further, the optional encryption was enabled with a randomly generated key, and the PAN ID was set to `6810'.

The communication follows the protocol described below. Communicated instructions are mapped to low level command, which are then executed on the Arduino. This is detailed in the following sections.

\subsection{Communication Protocol}

The communications protocol is formed of packets of the basic form: \\
\verb$<target><source><data><checksum>\r\n$.

\texttt{<target>} and \texttt{<source>} are the ASCII character `1' or `2' for the robots of group 11 and 12 accordingly, and the ASCII character `c' for the computer. For robot to computer messages, the targets `d' and `e' are also valid, used to send debug and error messages respectively.
\texttt{<data>} is the base64 encoded binary message, and \texttt{<checksum>} is a checksum of said message. The base64 encoding does not use padding bytes, but is otherwise standard.

The checksum consists of 2 characters, which are calculated as the checksum of all characters at even (starting 0) and odd indicies respectively, including \texttt{<target>} and \texttt{<source>}.
These checksums are calculated by retrieving the value of each the character in base64 encoding in the integer range 0-63, and XORing these together.
The resulting value from 0-63 is then encoded back into a base64 character.
For example, the checksum of `\texttt{dt6bas2}' is `\texttt{La}'.

All devices ignore all packets not addressed to it, as well as malformed packets. The validity of a packet is verified by the packet structure and the checksum.
Upon successful receipt of a packet, the recipient sends an acknowledgement to the sender.
An acknowledgement packet has the form `\verb%<target>$<checksum>\r\n%'. 
Acknowledgement packets are not acknowledged again.

The computer side will resend packets which have not been acknowledged.
The Arduino side does not verify packet receipt or resend packets, and discards packets which are longer than the internal buffer (60 bytes). 
To prevent flooding, only the last sent packet is resent. This suits the application well, as previously sent packets contain instructions to the robot which are superseded
by the more recent one.


\section{Arduino Architecture}

The robot is controlled by an on-board Arduino unit. This is programmed to
receive input from a native RF module, as well as the wheels rotary encoders.
Further, it controls the four movement motors, the single grabber motor, and
the power flow to the solenoid kicker. The Arduino main loops does three
things: It checks for new commands from the comms, it updates the rotary
encoder positions with the latest reading, and it updates its internal state
and the motor outputs. This is repeated until the Arduino is turned off.

\subsection{Low-Level Commands}

Internally, the Arduino operates using a set of low-level command. Examples
include opening the grabbers for a set amount of time, or moving to the right a
specified distance. Each command is a sequence of one or more bytes, the first
of which always identifies the command. The most significant bit of the first
byte indicates the interruptibility of the command, which we will touch on
later, with one indicating uninterruptibility, and zero indicating
interruptibility. The remaining 7 bits indicate the command itself, the
different possible values and their meaning are tabulated in \cref{llcmd}. As
with the comms ABI, depending on the command, a set number of argument bytes
will follow.

Low-level commands correspond directly to a single action the robot should
currently be doing; in particular not a sequence of actions. For example, the
grabbers opening, or the solenoid kicking. They typically have clearly defined
Arduino outputs (spinning the grabber motors, and enabling the solenoid in the
previous examples), as well as clear completion criteria, such as time elapsed
or distance travelled. Once a command is completed, the Arduino commences the
next low-level command immediately.

\subsection{Comms Input}

If the RFs internal buffer contains a valid sequence of new instructions
(\cref{comms}), this is translated into more low-level instructions which can
be used by the Arduino directly, and the buffer is cleared. If the RFs internal
buffer is \textit{not} a valid sequence of new instructions, but contains a new
line, it is considered to be a malformed packet, and the buffer is also
cleared.

The translation to low-level commands occurs immediately upon receiving new
instructions, and involves mapping each of the ABI commands to one or more
low-level commands. For more detail of what low-level commands are generated
for each ABI command, see \cref{abi}. The new string of low-level commands
effectively replace any in the current buffer, with a few notable exceptions.
Some commands (e.g. kicking or the grabbers opening), are uninterruptible,
which means that once commenced, they will complete before another command is
started. Since these also compile down into a sequence of low-level commands,
this sequence should not be interrupted. This is achieved with the
`uninterruptible' flag bit. If the first low-level command is uninterruptible,
the newly to insert commands do not replace it, but instead the first
interruptible command in the list.

\subsection{Distance}

For multiple of the low-level commands, the distance the robot has travelled,
or the angle it has rotated needs to be known. This is calculated from the
values taken off the rotary encoders on each wheel, which indicate how far
these wheels have rotated. In practice, due to the inconsistencies such as the
wheels not stopping immediately and turning at differing powers, these values
are not completely linear with the distance travelled or the angle rotated.
Instead, a non-linear approximation function was created from measured values
to translate rotary encoder positions into workable values for distance
travelled or angle rotated. This function uses measured values for several
points, and is linear between them.

\subsection{Auto-Correction}

Due to the motors on different sides having different powers, and asymmetry in
the design, without any corrections the robot drifted to one side. To
compensate for this, the motor powers are updated every time the main loop is
run, and motors whose rotary encoders show that they are ahead of the rest are
throttled. In practice, this approach introduces a slight jerkiness and
unpredictability to the movement, however this was deemed an acceptable
trade-off for straight movement.


\section{Introduction}

The aim of this project was to create a robot that would play a simplified version of 2-a-side football. We were recommended to use the provided Arduino Uno (Xino RF) board and LEGO pieces, a vision feed of the testing area (pitch), and computers to connect the two and implement any further logic.

This report will examine the decisions made in the design of each part and their impacts. Furthermore it will explain the principles our work relies on.


\section{Hardware Notes}

Grabbing the ball effectively is a fine balance of speed. Moving the grabbers
too quickly kicks the ball away. On the other hand, moving too slowly may lead
to missing the ball. Further, the design of the robot means that the ball must
be pushed in sharply to reset the kicker. The balance between these that was
found, is to close the grabbers at a moderate speed, then rapidly open them
part way and rapidly close them again. This resets the kicker, does not kick the ball away in most cases, and is fast enough to be usable.

Kicking the ball is a matter of speed, the faster we can kick, the less time
our opponents have to react. However, in order to kick, the ball must also be
positioned precicely in front of the kicker. To achieve this, the grabbers
rapidly push the ball into position again, correcting any drift from movement.
They open part-way, just enough for the kicker to kick without the ball
hitting the grabbers. Finally, the grabbers finish opening.

Spinning and moving tends to move the grabbers about. To avoid this, the
grabbers are powered slightly during both of these actions. If the grabbers are
open, they are constantly opening while the robot is moving, albeit very
slowly. Likewise, the grabbers keep closing if they are already closed. For
this purpose, the robot maintains a local state variable for the grabber. It is
worth stating that this is not measured or sensed, so manually moving the
grabbers will confuse the robot.

\section{Vision}

\subsection{Requirements}
You'll need the following python packages to successfully run the vision:

\begin{description}
\item \texttt{Polygon2} Polygon is a python package that handles polygonal shapes in 2D. 
\item \texttt{argparse} Python command-line parsing library
\item \texttt{pyserial} Python Serial Port Extension
\item \texttt{numpy} Array processing for numbers, strings, records, and objects.
\item \texttt{openCV} OpenCV-Python is the Python API of OpenCV. It combines the best qualities of OpenCV C++ API and Python language.
\end{description}

To install them run these commands in the terminal:

\shellcmd{pip install --user Polygon2==2.0.6} \\
\shellcmd{pip install --user argparse==1.3.0} \\
\shellcmd{pip install --user pyserial==2.7} \\
\shellcmd{pip install --user numpy}  

You can also learn how to install openCV from the following link:
\url{http://docs.opencv.org/2.4/doc/tutorials/introduction/linux_install/linux_install.html}

\subsection{Usage}
Before using the vision system, you can have a look at the vision feed by typing "xawtv" in the command prompt. This will launch only the vision feed, where you can experiment with the different settings.

In order to launch our vision system, you'll need to run the main vision file (python vision.py). At first, a window for the automatic colour calibration will pop out. You'll need to follow the instructions as printed in the terminal. The calibration goes through all the colours that are used for the vision (‘red’, ‘yellow’, ‘blue’, ‘green’, ‘pink’) and requires multiple clicks for each one of them to get their thresholds. You need to press the ‘q’ button after each calibrated colour. If you want to skip the calibration you can simply press the ‘Esc’ key and the vision will use the previously saved calibrations.

After this, the vision will be launched. There will be a window named ‘Filter output’, where you can see the vision feed with objects drawn on it representing the robots and the ball (if they are found). Robots will be represented by an inner circle for the team colour (‘yellow’ or ‘blue’), an outer circle identifying which of the two teammates it is (‘pink’ or ‘green’) and an arrow giving the direction of the robot.  The red ball is simply shown by drawing a red circle around it. You’ll also notice two other windows containing several trackbars. These trackbars are for filters that you can add to the output feed. There are filters to show only specific colours, or to different effects to the frame.

When the vision is running, it will provide the coordinates of all found objects, their orientation and velocity relative to the previous taken frame. These objects are returned as a dictionary and are passed onto the planner.



\section{Planning}

\subsection{Design}

The planner uses a system of goals and actions. Goals define an overall strategy in a given situation leading to a particular aim. Actions define one instruction given to the robot with preconditions for its execution. Goals are composed of a sequence of actions. A goal object selects the next action required to achieve its aim by traversing its ordered list of actions and evaluating their preconditions.

The planner makes its decision based only on the world state passed to it. This is described by an instance of the \emph{World} class passed to the planner. This object's state is updated by passing a new set of positions (as a dictionary of vectors) to the \texttt{update\_positions} method of \emph{World}. The \emph{World} class describes the state of the world from the vision, including position vectors for the robots, the ball and the goals. The \emph{World} class and its associated classes also provide methods on their data providing the planner with information about the world. Further utility functionality can be found in \emph{utils.py}.

The overall planner is used by calling \texttt{plan\_and\_act(world)}. This selects a goal based on the given world state  using its \texttt{get\_goal()} method. From this goal an action is generated using \texttt{generate\_action()}. The method then runs (using \texttt{actuate(action)} and returns a delay giving the time until the planner should be run again.

\subsection{Implementation}


Figure \ref{fig:receiveandpass} shows an example of Defender's goal \emph{ReceivingPass}. It shows a list of actions that are in a queue starting from the top. When action's preconditions are met, the action is executed and the process finishes. If any of the precondition is false, the next action in the queue is processed.

\begin{center}

\begin{figure}[H]
\centering
\begin{tabular}{ | l | p{110mm} | }
\hline
\textbf{Action} & \textbf{Preconditions} \\ \hline
GrabBall &
\begin{compactitem}
\item Defender can catch ball
\item Defender's grabbers are open
\end{compactitem}  \\ \hline

GoToStaticBall &
\begin{compactitem}
\item Defender is facing the catch point
\item Defender's grabbers are open
\item Ball is static
\end{compactitem}  \\ \hline

TurnToCatchPoint & 
\begin{compactitem}
\item Defender's grabbers are open
\end{compactitem}  \\ \hline

WaitForBallToCome &
\begin{compactitem}
\item Defender is facing ball
\item The ball can reach the robot
\item Defender can reach the ball
\item Grabbers are open
\item The ball is moving
\end{compactitem}  \\ \hline

FollowBall &
\begin{compactitem}
\item Defender's grabbers are open
\item The ball is moving
\end{compactitem}  \\ \hline

OpenGrabbers & 
\begin{compactitem}
\item Defender's grabbers are closed
\end{compactitem}  \\ \hline

FaceFriendly & 
\begin{compactitem}
\item Friendly is on the pitch
\end{compactitem}  \\ \hline
\end{tabular}
\par
\bigskip

\caption{ReceivingAndPass goal in the defender's planner. Shows actions and their corresponding preconditions that have to be satisfied.}

	\label{fig:receiveandpass}
\end{figure}



\end{center}

\subsection{Extensability}

Extending the planner is simply a case of adding goals and actions then adding the logic to select these in \texttt{select\_goal(world)}. Any new goals or actions should subclass Goal and Action respectively and override methods were stated.

In actions, the logic for performing should be placed in \texttt{perform(comms)}. This method is passed a \texttt{CommsManager} object through which the robot can be sent instructions. No more than one call should be made to this object in any given action. A new action should also override \texttt{get\_delay} giving an appropriate delay (in seconds) before the planner should run again. New actions can also have preconditions defined in a variable \texttt{preconditions}.

In goals, it may only be neccessary to write a Goal subclass with an ordered list of actions (from last to first). Otherwise the \texttt{generate\_action()} method can be overriden but similar logic should be followed.

\subsection{Integration}

An instance of the \texttt{Planner} is kept by the ``main.py'' script. Based on the delays given by the planner, it calls the planner at varying intervals, passing it the latest world model provided by the vision. The planner uses a \texttt{CommsManager} object to make control the robot.

\pagebreak
\appendix

\newcommand{\abispec}[6]{
\subsection{\texttt{#1}}

\begin{tabularx}{\textwidth}{>{\hsize=.5\hsize}XX}
\textbf{Instruction byte:}               & \texttt{#2} \\
\textbf{Argument bytes:}                 & #3 \\
\textbf{Argument type(s):}               & \texttt{#4} \\
\textbf{Action:}                         & #5 \\
\textbf{Relation to low-level commands:} & #6 \\
\end{tabularx}
}

\section{List of ABI Commands} \label{abi}

This section will outline the available instructions for the communications
ABI, the arguments they take, and the effect they will produce. A note on the
notation used for arguments: The arguments are standard C types, and are read
from the ABI byte buffer as this type directly. Since the Arduino is
little-endian, this for example means that \texttt{0x03 29}, read as a
\texttt{uint16\_t}, is 10499.

\abispec{WAIT}{0x00}{2}{uint16\_t}{Waits for the given time in milliseconds.}
    {Maps one-to-one to the low-level WAIT command.}
\abispec{BRAKE}{0x01}{2}{uint16\_t}
    {Brakes all motors for the given time in milliseconds.}
    {Relation to low-level commands: Maps one-to-one to the low-level
     \texttt{BRAKE} command.}
\abispec{STRAIT}{0x02}{2}{int16\_t}
    {Moves right the given distance in millimeters. This distance may be
     negative, in which case the robot moves left instead.}
    {Relation to low-level commands: Maps to an equivalent low-level
     \texttt{STRAIT} command, followed by a low-level \texttt{BRAKE} command.}
\abispec{SPIN}{0x03}{2}{int16\_t}
    {Rotates clockwise on the spot for the given number of minutes (60 minutes
     corresponds to one degree). This may be negative, in which case the robot
     rotates counter-clockwise instead..}
    {Maps to an equivalent low-level \texttt{SPIN} command, followed by a
     low-level \texttt{BRAKE} command.}
\abispec{KICK}{0x04}{2}{uint16\_t}
    {Opens the grabbers and actuates the kicker for the given time in
     milliseconds.}
    {First, a \texttt{GRABBER\_FORCE} is issued to ensure the ball is in front
     of a kicker. Then, a part-way \texttt{GRABBER\_OPEN} is done, followed by
     the equivalent low-level \texttt{KICK}. Finally, a \texttt{GRABBER\_OPEN}
     is issued to finish opening the grabbers. All of these are
     uninterruptible.}
\abispec{GRABBER\_OPEN}{0x06}{0}{\textnormal{\textit{none}}}{Opens the grabbers.}
    {Maps one-to-one to an uninterruptible low-level \texttt{GRABBER\_OPEN}.}
\abispec{GRABBER\_CLOSE}{0x07}{0}{\textnormal{\textit{none}}}
    {Closes the grabbers in multiple stages to ensure the ball is caught well.}
    {First, a \texttt{GRABBER\_CLOSE} is issued to get the ball close to the
     solenoid.  Then, a \texttt{GRABBER\_OPEN}, followed by a
     \texttt{GRABBER\_FORCE} is issued, to ensure that the ball presses the
     solenoid in. Finally, since the \texttt{GRABBER\_FORCE}'s power leads to
     the grabbers bouncing back to being partially open, a
     \texttt{GRABBER\_CLOSE} is issued.}


\end{document}
