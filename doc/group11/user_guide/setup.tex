\section{Setup}
\subsection{Requirements}

In order to run the computer side control program, the following python
packages need to be installed:

\begin{itemize}
\item \texttt{Polygon2}, for handling interactions of 2D polygons.
\item \texttt{pyserial}, for interacting with the serial RF device.
\item \texttt{numpy}, for various numerical calculations.
\item \texttt{openCV}, for vision processing.
\end{itemize}

On the university DICE machines, \texttt{numpy} and \texttt{openCV} are already
preinstalled. To install the remaining packages, run the following commands:

\shellcmd{pip install --user Polygon2==2.0.6} \\
\shellcmd{pip install --user pyserial==2.7} \\

\subsection{Arduino}

The Arduino must be programmed with the robot's firmware before use. If this is
not already the case, it can be programmed by running the \texttt{arduino}
program, and opening the \texttt{/arduino/group11/group11.ino} sketch. If not
the case already, the sketchbook location should be set to \texttt{/arduino/},
to ensure that the libraries used are found. Next, the Arduino should be
connected to the computer, and the sketch compiled and uploaded to it using the
button for this purpose. The USB cable should then be removed. Finally, if the
robot was on during this process, it should be briefly switched off to allow
the Arduino serial device to reset.
