\documentclass[a4paper,12pt]{scrartcl}
\usepackage{tabularx}
\usepackage[pdfusetitle, hidelinks]{hyperref}
\newcommand{\shellcmd}[1]{\indent\indent\texttt{\footnotesize\$ #1}}
\author{Thomas Kerber, Marek Strelec, Euan Hunter}
\title{SDP User Guide}
\subtitle{Group 11}

\begin{document}
\maketitle

\section{Setup}
\subsection{Requirements}

In order to run the computer side control program, the following python
packages need to be installed:

\begin{itemize}
\item \texttt{Polygon2}, for handling interactions of 2D polygons.
\item \texttt{pyserial}, for interacting with the serial RF device.
\item \texttt{numpy}, for various numerical calculations.
\item \texttt{openCV}, for vision processing.
\end{itemize}

On the university DICE machines, \texttt{numpy} and \texttt{openCV} are already
preinstalled. To install the remaining packages, run the following commands:

\shellcmd{pip install --user Polygon2==2.0.6} \\
\shellcmd{pip install --user pyserial==2.7} \\

\subsection{Arduino}

The Arduino must be programmed with the robot's firmware before use. If this is
not already the case, it can be programmed by running the \texttt{arduino}
program, and opening the \texttt{/arduino/group11/group11.ino} sketch. If not
the case already, the sketchbook location should be set to \texttt{/arduino/},
to ensure that the libraries used are found. Next, the Arduino should be
connected to the computer, and the sketch compiled and uploaded to it using the
button for this purpose. The USB cable should then be removed. Finally, if the
robot was on during this process, it should be briefly switched off to allow
the Arduino serial device to reset.

\section{Robot Design}

The robot is powered by four NXT motors at the sides. These are geared up, and
each drive one of the sideways facing holonomic wheels. Holonomic wheels are
used instead of ordinary wheels to allow for smooth on the spot spinning, which
normal wheels are not capable of, as all four do not reside on the same turning
circle. The motors are connected through a spine at the bottom of the robot,
on top of which the solenoid kicker rests, facing forward. The solenoid kicker
does not use a spring to reset itself, and instead relies on the ball being
pushed into it to reset it. The front side of the robot has an old style LEGO
motor on both the left and right sides, each to move a lightweight, slightly
curved grabber arm. Both of these close at the front, and tuck away to the
sides when not used. They are mounted at slightly different heights to allow
them to close fully without colliding.

On the inside of the robot there is some space created by the four movement
motors. This is occupied by a rotary encoder board, which receives input from
the movement motors, the Arduino, along with the supplied motor and power
boards, and various wires and smaller components. These include the wires from
each of the motors, the boards splitting the power and rotary encoder cables
for the NXT motors, and a breadboard regulating the power input for the
solenoid kicker. Both this breadboard, and the power board are wired together,
and both draw power for a battery pack containing 4 lithium-ion batteries
mounted at the rear of the robot. At the base of the battery pack a power
switch is located. A frame is around the loose parts of the robot, and
continues up to the top of the cabling, with beams running above the Arduino to
mount the top plate on.

\section{Hardware Notes}

Grabbing the ball effectively is a fine balance of speed. Moving the grabbers
too quickly kicks the ball away. On the other hand, moving too slowly may lead
to missing the ball. Further, the design of the robot means that the ball must
be pushed in sharply to reset the kicker. The balance between these that was
found, is to close the grabbers at a moderate speed, then rapidly open them
part way and rapidly close them again. This resets the kicker, does not kick
the ball away in most cases, and is fast enough to be usable.

Kicking the ball is a matter of speed, the faster we can kick, the less time
our opponents have to react. However, in order to kick, the ball must also be
positioned precisely in front of the kicker. To achieve this, the grabbers
rapidly push the ball into position again, correcting any drift from movement.
They open part-way, just enough for the kicker to kick without the ball
hitting the grabbers. Finally, the grabbers finish opening.

Spinning and moving tends to move the grabbers about. To avoid this, the
grabbers are powered slightly during both of these actions. If the grabbers are
open, they are constantly opening while the robot is moving, albeit very
slowly. Likewise, the grabbers keep closing if they are already closed. For
this purpose, the robot maintains a local state variable for the grabber. It is
worth stating that this is not measured or sensed, so manually moving the
grabbers will confuse the robot. On startup, the robot assumes the grabbers to
be in their default, open position.

\section{Usage}

Before use, ensure that the four lithium-ion batteries are fully charged. Our
group shares the charger with group 12. Ensure that the batteries are properly
inserted into the two battery holders at the rear of the robot. To turn on the
robot, flick the switch at the base of the battery packs. A red light will be
visible from the Arduino, indicating that it is powered.

Our robot is operated together with that of group 12, it is assumed that their
robot has also been switched on at this point. To begin with, the device paths
of both groups USB RF sticks must be determined. To determine these, first
insert the RF stick for group 11. Then run \texttt{ls /dev/ttyACM*} to list all
serial USB devices availabl. This should return only one path, which can be
noted as the path for group 11's RF stick. Group 12's device path can then be
determined by inserting their RF stick and re-running \texttt{ls /dev/ttyACM*}.
One new path should be among those returned, which can be noted as the path to
group 12's RF stick.

Further information required to run the control unit is: The room being played
in (3.D03 or 3.D04), which side of the pitch, left or right, is ours (viewed
from the door), and which color our team will have (blue or yellow). Armed with
this knowledge, make sure that a top plate of each colour, as well as the ball,
are on the pitch. Then, \texttt{cd} into the project root directory, and
execute the following command:

\shellcmd{./main -p game -1 <PATH1> -2 <PATH2> -c <COLOUR> --pitch <NUM> -g <SIDE>}

\noindent Using the following values for the arguments:

\begin{itemize}
    \item The path of group 11's RF stick for \texttt{<PATH1>}.
    \item The path of group 12's RF stick for \texttt{<PATH2>}.
    \item `\texttt{y}' or `\texttt{b}' for \texttt{<COLOUR>}, if our team is
        yellow or blue respectivey.
    \item `\texttt{0}' or `\texttt{1}' for \texttt{<NUM>}, if the game is being
        played in room 3.D03 or 3.D04 respectively.
    \item `\texttt{left}' or `\texttt{right}' for \texttt{<SIDE>}, if our goal
        is on the left or the right side of the pitch respectively.
\end{itemize}

A few windows should open up, the most prominent one of which displays a still
of the pitch for calibration.

\subsection{Calibration}

On the still of the pitch all four top plates and the ball should be clearly
visible as positioned earlier. The terminal will prompt the operator to click
on samples of red, blue, yellow, green, and pink in turn, for the ball and the
top-plate parts of the respective colours. This calibrates for the colours,
ensuring that they are accurately detected. Of each colour, multiple samples
should be clicked, including all places it is visible on the pitch (i.e. the
green circles of all top plates, and not just one should be sampled). After
several samples of each colour have been clicked, press the `q' key to move on
to the next. The colours are queried in the order listed above, and a request
for samples of the current colour is also printed to the terminal. Once
calibrations are complete, they are saved locally. Due to changing light
conditions, you should recalibrate at least every few hours, however it is
possible to skip the calibration phase and use the saved values by pressing
escape at the start of calibration.

\subsection{Online Usage}

After the vision is lunched, there will be a window named \texttt{Filter}
output, where you can see the vision feed with objects drawn on it representing
the robots and the ball (if they are found).  Robots will be represented by an
inner circle for the team colour (yellow or blue), an outer circle identifying
which of the two team mates it is (pink or green, referring to the colour of
the majority of circles on the top plate) and an arrow giving the orientation
of the robot. The red ball is simply shown by drawing a red circle around it.
Further, two other windows will be open, containing several trackbars.  These
trackbars are for filters that you can add to the output feed. There are
filters to show only specific colours, or to different effects to the frame.
For day to day usage, these do not need to be used.

While the control program is running, the overall strategy and logging can be
modified. Entering `\texttt{debug}', `\texttt{info}', `\texttt{warn}', or
`\texttt{error}', and pressing enter, sets the logging level appropriately.
Further, a window to select the game state will have opened, allowing selection
from `\texttt{kickoff-them}', `\texttt{kickoff-us}', `\texttt{penalty-them}',
`\texttt{penalty-us}', `\texttt{normal-play}', and `\texttt{stop}'. Pressing
any of these sets the game to the respective state. At startup, the state is
`\texttt{stop}', and the robots should remain idle. Once the referee orders a
kickoff, either `\texttt{kickoff-them}', or `\texttt{kickoff-us}' should be
pressed, depending on if the opposing team or our team has kickoff
respectively. Likewise, should the referee call for a penalty kick, either
`\texttt{penalty-them}' or `\texttt{penalty-us}' should be pressed, depending
on if the opposing team or our team take the penalty respectively. The
`\texttt{stop}' button can be pressed whenever the referee stops the game.
Although it should not have to be used typically, pressing
`\texttt{normal-play}' skips the kickoff phase of play, and immediately begins
playing. This may be useful if the program had to be restarted mid-match.

At halftime, both teams switch sides. The control program needs to be informed
of the switch, either by typing in `\texttt{switch}' into the console, and
pressing enter, or by closing the control program and re-running it with the
other pitch side. To terminate the control program, press `Ctrl-D' with the
terminal window selected. If this fails, press enter and try again.


\section{Troubleshooting}

Below are the actions to take for various common problems which may occur. The
solutions are not guaranteed to work, but have in experience been effective.

\subsection{The Robot Doesn't Move}

If the robot is not moving, the first thing to do is to check the batteries.
Ensure that they are present, and charged. Further, confirm that the robot was
switched on. If the Arduino has a red light on, it is receiving power and the
issue is more likely to be one of communications. In this case switch the robot
off for a few seconds and turn it on again to reset the serial device.

\subsection{The Control Program Doesn't Start}

The control program not starting is likely due to the RF control signals
blocking. To test this, run the control program without the argument for the RF
stick. If it runs, then try waiting for five seconds, then re-running with the
normal arguments. If after a few attempts, this also does not work, then double
check the guard characters set on the RF stick. They should be three tilde
characters (`\verb$~~~$'). The proprietary CISCO tools located in the
`\texttt{/tools}' folder can be used to check this, and, if needed, to reset
them.

\subsection{The Vision is Misdetecting Objects}

The vision misdetecting objects is likely due to miscalibration. Close the
controller, restart and recalibrate. Take care to select a wide range of colour
samples, including the fringes of the circles to detect.

\subsection{The Robot is Very Slow}

The batteries are low. Recharge or replace them.

\subsection{The Vision is Solid Green}

The video input is set to draw from the wrong source. Close the control
program, and run `\texttt{xawtv}'. In the options, select svideo as the source.

\subsection{The Kicker Doesn't Work}

Double check the I/O port connections with the kicker. Try also listening at
the breadboard during a kick for an audible click. If this can be heard, the
issue is in the wiring from the batteries to the breadboard, or from this to
the kicker. If not, the issue is either the connectors being plugged into the
wrong I/O port (it should be the furthest inward on the I/O board), or a faulty
connection from the I/O port to the breadboard.

\end{document}
