\section{Usage}

Before use, ensure that the four lithium-ion batteries are fully charged. Our
group shares the charger with group 12. Ensure that the batteries are properly
inserted into the two battery holders at the rear of the robot. To turn on the
robot, flick the switch at the base of the battery packs. A red light will be
visible from the Arduino, indicating that it is powered.

Our robot is operated together with that of group 12, it is assumed that their
robot has also been switched on at this point. To begin with, the device paths
of both groups USB RF sticks must be determined. To determine these, first
insert the RF stick for group 11. Then run \texttt{ls /dev/ttyACM*} to list all
serial USB devices availabl. This should return only one path, which can be
noted as the path for group 11's RF stick. Group 12's device path can then be
determined by inserting their RF stick and re-running \texttt{ls /dev/ttyACM*}.
One new path should be among those returned, which can be noted as the path to
group 12's RF stick.

Further information required to run the control unit is: The room being played
in (3.D03 or 3.D04), which side of the pitch, left or right, is ours (viewed
from the door), and which color our team will have (blue or yellow). Armed with
this knowledge, make sure that a top plate of each colour, as well as the ball,
are on the pitch. Then, \texttt{cd} into the project root directory, and
execute the following command:

\shellcmd{./main -p game -1 <PATH1> -2 <PATH2> -c <COLOUR> --pitch <NUM> -g <SIDE>}

\noindent Using the following values for the arguments:

\begin{itemize}
    \item The path of group 11's RF stick for \texttt{<PATH1>}.
    \item The path of group 12's RF stick for \texttt{<PATH2>}.
    \item `\texttt{y}' or `\texttt{b}' for \texttt{<COLOUR>}, if our team is
        yellow or blue respectivey.
    \item `\texttt{0}' or `\texttt{1}' for \texttt{<NUM>}, if the game is being
        played in room 3.D03 or 3.D04 respectively.
    \item `\texttt{left}' or `\texttt{right}' for \texttt{<SIDE>}, if our goal
        is on the left or the right side of the pitch respectively.
\end{itemize}

A few windows should open up, the most prominent one of which displays a still
of the pitch for calibration.

\subsection{Calibration}

Before using the vision system, you can have a look at the vision feed by typing \emph{xawtv} in the command prompt. This will launch only the vision feed, where you can experiment with the different settings.

When the vision starts, it will enter calibration mode. In this mode, a still
of the pitch is displayed. In turn, red, blue, yellow, green, and pink will be
calibrated for. For each of these, click multiple times on parts of the still
which should be recognised as the specified color. When done with a color,
press `\texttt{q}' to move on to the next.

The calibrations will be saved locally, and can be loaded by pressing escape
when the calibration mode starts.

\subsection{Online Usage}

After the vision is lunched, there will be a window named \texttt{Filter} output, where you can see the vision feed with objects drawn on it representing the robots and the ball (if they are found).
Robots will be represented by an inner circle for the team colour (\texttt{yellow} or \texttt{blue}), an outer circle identifying which of the two team mates it is (\texttt{pink} or \texttt{green}) and an arrow giving the direction of the robot. The red ball is simply shown by drawing a red circle around it. You'll also notice two other windows containing several trackbars.
These trackbars are for filters that you can add to the output feed. There are filters to show only specific colours, or to different effects to the frame.

While the control program is running, the overall strategy and logging can be
modified. Entering `\texttt{debug}', `\texttt{info}', `\texttt{warn}', or
`\texttt{error}', and pressing enter, sets the logging level appropriately.
Likewise, entering a plan name and pressing enter switches the running plan to
what was entered. Entering `\texttt{stop}' unsets the active plan and leaves
the robot idle.

\subsection{Plans}

A small number of plans are available to run:

\begin{itemize}
    \item `\texttt{move-grab}', to move to the ball and grab it.
    \item `\texttt{m1}', to do milestone 3, task 1.
    \item `\texttt{m2}', to do milestone 3, task 2.
    \item `\texttt{m31}', to do milestone 3, task 3.1.
    \item `\texttt{m32}', to do milestone 3, task 3.2.
\end{itemize}
