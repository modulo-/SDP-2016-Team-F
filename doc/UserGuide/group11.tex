\documentclass[a4paper]{scrartcl}
\usepackage{tabularx}
\author{Thomas Kerber}
\title{User Guide}
\subtitle{Group 11}

\begin{document}
\maketitle
\section{Setup}

To setup on DiCE, run `\texttt{pip install --user pyserial}'. Connect the
arduino to a USB port and run the \texttt{arduino} program. Open the file
\verb$arduino/group11/group11.ino$, and ensure that the libraries in
\verb$arduino/libraries/$ are loaded. Compile and upload the code to the
arduino. Remove the USB cable.

\section{Calibration}

The robot currently needs manual calibration. This involves tweaking the
scalars in the \texttt{rotation\_time\_from\_angle},
\texttt{move\_time\_from\_distance}, and \texttt{kick\_time\_from\_distance}
functions in \verb$src/group11cmd.py$, until they behave as expected. E.g. if
the `\texttt{spin\_cc 90}' makes the robot spin only 70 degrees, the scalar in
\texttt{rotation\_time\_from\_angle} has to be increased.

\section{Usage}

Once ready to use, attach the battery cable. Then calibrate the robot for the
battery change (this may be done beforehand, assuming the battery change does
not vary greatly).

Plug the RF stick in the computer, enter the `\verb$src/$' directory, and
execute `\texttt{./group11cmd.py <device path>}', where `\texttt{<device
path>}' is the device identified of the RF stick (e.g. `\verb$/dev/ttyACM0$').

A list of available commands will be shown. When entered, these are sent to the
robot, which executes them. For the most part these should be self-descriptive,
if more information is required, see the technical specifications section on
the command shell.

\end{document}
