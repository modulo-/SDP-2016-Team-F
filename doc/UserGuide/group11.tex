\documentclass[a4paper]{scrartcl}
\author{Thomas Kerber, Marek Strelec}
\title{User Guide}
\subtitle{Group 11}

\newcommand{\shellcmd}[1]{\indent\indent\texttt{\footnotesize\# #1}}

\usepackage[pdfusetitle, hidelinks]{hyperref}
\usepackage{tabularx}

\begin{document}
\maketitle
\section{Setup}
\subsection{requirements}
You'll need the following python packages to successfully run on DICE machine:

\begin{itemize}
\item \texttt{Polygon2} Polygon is a python package that handles polygonal shapes in 2D. 
\item \texttt{argparse} Python command-line parsing library
\item \texttt{pyserial} Python Serial Port Extension
\item \texttt{numpy} Array processing for numbers, strings, records, and objects.
\item \texttt{openCV} OpenCV-Python is the Python API of OpenCV. It combines the best qualities of OpenCV C++ API and Python language.
\end{itemize}

To install them run these commands in the terminal:

\shellcmd{pip install --user Polygon2==2.0.6} \\
\shellcmd{pip install --user argparse==1.3.0} \\
\shellcmd{pip install --user pyserial==2.7} \\
\shellcmd{pip install --user numpy}  

You can also learn how to install openCV from the following link:
\url{http://docs.opencv.org/2.4/doc/tutorials/introduction/linux_install/linux_install.html}

\subsection{arduino}
Connect the arduino to a USB port and run the \texttt{arduino} program. Open the file \verb$arduino/group11/group11.ino$, and ensure that the libraries in
\verb$arduino/libraries/$ are loaded. Compile and upload the code to the
arduino. Remove the USB cable.


\section{Usage}

Once ready to use, place a charged 8-pack of batteries in the back holder, and
connect it to the arduino power cable. Place 2 charges 4-packs of batteries in
the other two back holders, and attack the remaining power cables to them.
These two packs power the kicker.

Plug the RF stick in the computer, ensure top plates of all types, as well as
the ball, are on the pitch, and execute
`\texttt{./main -p <PLAN> -1 <PATH> -c <COLOR>}', there \texttt{<PLAN>} is the
plan to run (see section below), \texttt{<PATH>} is the device path of the RF
stick (e.g. `\verb$/dev/ttyACM0$'), and \texttt{<COLOR>} is the team color,
which must be one of `\texttt{blue}', `\texttt{b}', `\texttt{yellow}', or
`\texttt{y}' respectively. Further options notably include the `\texttt{-l}'
option, which sets the logging level. E.g. `\texttt{-l info}' enables info
messages in the logger.

\subsection{Calibration}

Before using the vision system, you can have a look at the vision feed by typing \emph{xawtv} in the command prompt. This will launch only the vision feed, where you can experiment with the different settings.

When the vision starts, it will enter calibration mode. In this mode, a still
of the pitch is displayed. In turn, red, blue, yellow, green, and pink will be
calibrated for. For each of these, click multiple times on parts of the still
which should be recognised as the specified color. When done with a color,
press `\texttt{q}' to move on to the next.

The calibrations will be saved locally, and can be loaded by pressing escape
when the calibration mode starts.

\subsection{Online Usage}

After the vision is lunched, there will be a window named \texttt{Filter} output, where you can see the vision feed with objects drawn on it representing the robots and the ball (if they are found).
Robots will be represented by an inner circle for the team colour (\texttt{yellow} or \texttt{blue}), an outer circle identifying which of the two team mates it is (\texttt{pink} or \texttt{green}) and an arrow giving the direction of the robot. The red ball is simply shown by drawing a red circle around it. You'll also notice two other windows containing several trackbars.
These trackbars are for filters that you can add to the output feed. There are filters to show only specific colours, or to different effects to the frame.

While the control program is running, the overall strategy and logging can be
modified. Entering `\texttt{debug}', `\texttt{info}', `\texttt{warn}', or
`\texttt{error}', and pressing enter, sets the logging level appropriately.
Likewise, entering a plan name and pressing enter switches the running plan to
what was entered. Entering `\texttt{stop}' unsets the active plan and leaves
the robot idle.

\pagebreak

\subsection{Plans}

A small number of plans are available to run:

\begin{itemize}
    \item `\texttt{move-grab}', to move to the ball and grab it.
    \item `\texttt{m1}', to do milestone 3, task 1.
    \item `\texttt{m2}', to do milestone 3, task 2.
    \item `\texttt{m31}', to do milestone 3, task 3.1.
    \item `\texttt{m32}', to do milestone 3, task 3.2.
\end{itemize}

\end{document}
