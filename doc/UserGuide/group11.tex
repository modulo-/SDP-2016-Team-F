\documentclass[a4paper]{scrartcl}
\usepackage{tabularx}
\author{Thomas Kerber}
\title{User Guide}
\subtitle{Group 11}

\begin{document}
\maketitle
\section{Setup}

To setup on DiCE, run `\texttt{pip install --user pyserial}'. Connect the
arduino to a USB port and run the \texttt{arduino} program. Open the file
\verb$arduino/group11/group11.ino$, and ensure that the libraries in
\verb$arduino/libraries/$ are loaded. Compile and upload the code to the
arduino. Remove the USB cable.

\section{Calibration}

The robot currently needs manual calibration. This involves tweaking the
scalars in the \texttt{rotation\_time\_from\_angle},
\texttt{move\_time\_from\_distance}, and \texttt{kick\_time\_from\_distance}
functions in \verb$src/group11cmd.py$, until they behave as expected. E.g. if
the `\texttt{spin\_cc 90}' makes the robot spin only 70 degrees, the scalar in
\texttt{rotation\_time\_from\_angle} has to be increased.

\section{Usage}

Once ready to use, attach the battery cable. Then calibrate the robot for the
battery change (this may be done beforehand, assuming the battery change does
not vary greatly).

Plug the RF stick in the computer, enter the `\verb$src/$' directory, and
execute `\texttt{./group11cmd.py <device path>}', where `\texttt{<device
path>}' is the device identified of the RF stick (e.g. `\verb$/dev/ttyACM0$').

A list of available commands will be shown. When entered, these are sent to the
robot, which executes them. For the most part these should be self-descriptive,
if more information is required, see the technical specifications section on
the command shell.

\section{Vision}

\subsection{Requirements}
You'll need the following python packages to successfully run the vision:

\begin{description}
\item \texttt{Polygon2} Polygon is a python package that handles polygonal shapes in 2D. 
\item \texttt{argparse} Python command-line parsing library
\item \texttt{pyserial} Python Serial Port Extension
\item \texttt{numpy} Array processing for numbers, strings, records, and objects.
\item \texttt{openCV} OpenCV-Python is the Python API of OpenCV. It combines the best qualities of OpenCV C++ API and Python language.
\end{description}

To install them run these commands in the terminal:

\shellcmd{pip install --user Polygon2==2.0.6} \\
\shellcmd{pip install --user argparse==1.3.0} \\
\shellcmd{pip install --user pyserial==2.7} \\
\shellcmd{pip install --user numpy}  

You can also learn how to install openCV from the following link:
\url{http://docs.opencv.org/2.4/doc/tutorials/introduction/linux_install/linux_install.html}

\subsection{Usage}
Before using the vision system, you can have a look at the vision feed by typing "xawtv" in the command prompt. This will launch only the vision feed, where you can experiment with the different settings.

In order to launch our vision system, you'll need to run the main vision file (python vision.py). At first, a window for the automatic colour calibration will pop out. You'll need to follow the instructions as printed in the terminal. The calibration goes through all the colours that are used for the vision (‘red’, ‘yellow’, ‘blue’, ‘green’, ‘pink’) and requires multiple clicks for each one of them to get their thresholds. You need to press the ‘q’ button after each calibrated colour. If you want to skip the calibration you can simply press the ‘Esc’ key and the vision will use the previously saved calibrations.

After this, the vision will be launched. There will be a window named ‘Filter output’, where you can see the vision feed with objects drawn on it representing the robots and the ball (if they are found). Robots will be represented by an inner circle for the team colour (‘yellow’ or ‘blue’), an outer circle identifying which of the two teammates it is (‘pink’ or ‘green’) and an arrow giving the direction of the robot.  The red ball is simply shown by drawing a red circle around it. You’ll also notice two other windows containing several trackbars. These trackbars are for filters that you can add to the output feed. There are filters to show only specific colours, or to different effects to the frame.

When the vision is running, it will provide the coordinates of all found objects, their orientation and velocity relative to the previous taken frame. These objects are returned as a dictionary and are passed onto the planner.



\end{document}
