\documentclass[a4paper]{scrartcl}
\usepackage{tabularx}
\author{Thomas Kerber}
\title{User Guide}
\subtitle{Group 11}

\begin{document}
\maketitle
\section{Setup}

To setup on DiCE, run `\texttt{pip install --user pyserial polygon2}'. Connect
the arduino to a USB port and run the \texttt{arduino} program. Open the file
\verb$arduino/group11/group11.ino$, and ensure that the libraries in
\verb$arduino/libraries/$ are loaded. Compile and upload the code to the
arduino. Remove the USB cable.

\section{Usage}

Once ready to use, place a charged 8-pack of batteries in the back holder, and
connect it to the arduino power cable. Place 2 charges 4-packs of batteries in
the other two back holders, and attack the remaining power cables to them.
These two packs power the kicker.

Plug the RF stick in the computer, ensure top plates of all types, as well as
the ball, are on the pitch, and execute
`\texttt{./main -p <PLAN> -1 <PATH> -c <COLOR>}', there \texttt{<PLAN>} is the
plan to run (see section below), \texttt{<PATH>} is the device path of the RF
stick (e.g. `\verb$/dev/ttyACM0$'), and \texttt{<COLOR>} is the team color,
which must be one of `\texttt{blue}', `\texttt{b}', `\texttt{yellow}', or
`\texttt{y}' respectively. Further options notably include the `\texttt{-l}'
option, which sets the logging level. E.g. `\texttt{-l info}' enables info
messages in the logger.

\subsection{Calibration}

When the vision starts, it will enter calibration mode. In this mode, a still
of the pitch is displayed. In turn, red, blue, yellow, green, and pink will be
calibrated for. For each of these, click multiple times on parts of the still
which should be recognised as the specified color. When done with a color,
press `\texttt{q}' to move on to the next.

The calibrations will be saved locally, and can be loaded by pressing escape
when the calibration mode starts.

\subsection{Online Usage}

While the control program is running, the overall strategy and logging can be
modified. Entering `\texttt{debug}', `\texttt{info}', `\texttt{warn}', or
`\texttt{error}', and pressing enter, sets the logging level appropriately.
Likewise, entering a plan name and pressing enter switches the running plan to
what was entered. Entering `\texttt{stop}' unsets the active plan and leaves
the robot idle.

\subsection{Plans}

A small number of plans are available to run:

\begin{itemize}
    \item `\texttt{move-grab}', to move to the ball and grab it.
    \item `\texttt{m1}', to do milestone 3, task 1.
    \item `\texttt{m2}', to do milestone 3, task 2.
    \item `\texttt{m31}', to do milestone 3, task 3.1.
    \item `\texttt{m32}', to do milestone 3, task 3.2.
\end{itemize}

\end{document}
