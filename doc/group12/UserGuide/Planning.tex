\section{Planning}
\lstset{language=Bash, showstringspaces=false}

\subsection{Running the planner}

The planner can be run as part of the system from the command line. This is done using:

\begin{lstlisting}
python main.py [-2PATH] OPTIONS TEAM-COLOUR -g GOAL-END
\end{lstlisting}

where TEAM-COLOUR refers to the colour of the controlled team's top plate and can be either -y or -b (yellow or blue)
and OPTIONS can be --debug Set logging level to 'debug'
	   	   --info  Set logging level to 'info'
		   --warn  Set logging level to 'warn'
		   --error Set logging level to 'error'
and GOAL-END refers to the controlled team's goal end, either 'left' or 'right'

\subsection{Calibrating the planner}

The calculations for pitch geometry in the planner depend on calibrating certain constants in the planner code.

\begin{table}[H]
  \centering
  \begin{tabularx}{\textwidth}{l l l}
    \toprule
    \textbf{Name} & \textbf{file} & \textbf{Description \\
    \midrule
    ROBOT\_WIDTH, ROBOT\_LENGTH & world.py & Dimensions of a standard robot
    \midrule
    BALL\_WIDTH, BALL\_LENGTH & world.py & Dimensions of the ball
    \midrule
    GOAL\_WIDTH, GOAL\_LENGTH & world.py & Dimensions of the goals
    \midrule
    GOAL\_LOWER, GOAL\_HIGHER & world.py & Y value for topmost and bottommost points of goals
    \midrule
    [ROBOT].\_receiving\_area & main.py & Dimensions of area in which robot can grab ball
    \midrule
    ROTATION\_THRESHOLD, FACING\_ROTATION\_THRESHOLD & models\_attacker.py & Difference for angles in which robot can be considered to be facing a point
    \midrule
    GRAB\_DISTANCE & models\_attacker.py & Distance from robot's centre at which the robot should grab the ball
    \bottomrule
  \end{tabular}
\end{table}
