\section{Planning}

\subsection{Plans}

A small number of plans are available to run:

\begin{itemize}
    \item `\texttt{move-grab}', to move to the ball and grab it.
    \item `\texttt{m1}', to do milestone 3, task 1 (receiving a pass).
    \item `\texttt{m2}', to do milestone 3, task 2 (receiving, turning and passing).
    \item `\texttt{m31}', to do milestone 3, task 3.1 (intercepting a pass).
    \item `\texttt{m32}', to do milestone 3, task 3.2 (defending the goal).
\end{itemize}


\subsection{Calibrating the planner}

The calculations for pitch geometry in the planner depend on calibrating certain constants in the planner code.


\begin{table}[H]
\begin{tabularx}{\textwidth}{ XlX }
\toprule
\textbf{Name} & \textbf{File} & \textbf{Description} \\
\midrule
\verb|ROBOT_WIDTH|, \verb|ROBOT_LENGTH| & \texttt{world.py} & Dimensions of a standard robot. \\
\verb|BALL_WIDTH|, \verb|BALL_LENGTH| & \texttt{world.py} & Dimensions of the ball. \\
\verb|GOAL_WIDTH|, \verb|GOAL_LENGTH| & \texttt{world.py} & Dimensions of the goals. \\
\verb|GOAL_LOWER|, \verb|GOAL_HIGHER| & \texttt{world.py} & Y value for topmost and bottommost points of goals. \\
\verb|[ROBOT]._receiving_area| & \texttt{main.py} & Dimensions of area in which robot can grab ball. \\
\verb|ROTATION_THRESHOLD|, \verb|FACING_ROTATION_THRESHOLD| &\texttt{models\_attacker.py} & Difference for angles in which robot can be considered to be facing a point. \\
\verb|GRAB_DISTANCE| & \texttt{models\_attacker.py} & Distance from robot's centre at which the robot should grab the ball. \\
\bottomrule
\end{tabularx}
\caption{Planner Configurable Variables}
\end{table}
