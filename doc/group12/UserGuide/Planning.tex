\section{Planning}
\lstset{language=Bash, showstringspaces=false}

\subsection{Running the planner}

The planner can be run as part of the system from the command line. This is done using:

\begin{lstlisting}
python main.py [-2PATH] OPTIONS TEAM-COLOUR -g GOAL-END
\end{lstlisting}

where TEAM-COLOUR refers to the colour of the controlled team's top plate and can be either -y or -b (yellow or blue)
and OPTIONS can be --debug, Set logging level to 'debug' \\
	   	   --info,  Set logging level to 'info' \\
		   --warn,  Set logging level to 'warn' \\
		   --error, Set logging level to 'error' \\
and GOAL-END refers to the controlled team's goal end, either 'left' or 'right'

\subsection{Calibrating the planner}

The calculations for pitch geometry in the planner depend on calibrating certain constants in the planner code.


\begin{table}[H]
\begin{tabularx}{\textwidth}{ XlX }
\toprule
\textbf{Name} & \textbf{File} & \textbf{Description} \\
\midrule
\verb|ROBOT_WIDTH|, \verb|ROBOT_LENGTH| & \texttt{world.py} & Dimensions of a standard robot. \\
\verb|BALL_WIDTH|, \verb|BALL_LENGTH| & \texttt{world.py} & Dimensions of the ball. \\
\verb|GOAL_WIDTH|, \verb|GOAL_LENGTH| & \texttt{world.py} & Dimensions of the goals. \\
\verb|GOAL_LOWER|, \verb|GOAL_HIGHER| & \texttt{world.py} & Y value for topmost and bottommost points of goals. \\
\verb|[ROBOT]._receiving_area| & \texttt{main.py} & Dimensions of area in which robot can grab ball. \\
\verb|ROTATION_THRESHOLD|, \verb|FACING_ROTATION_THRESHOLD| &\texttt{models\_attacker.py} & Difference for angles in which robot can be considered to be facing a point. \\
\verb|GRAB_DISTANCE| & \texttt{models\_attacker.py} & Distance from robot's centre at which the robot should grab the ball. \\
\bottomrule
\end{tabularx}
\caption{Planner Configurable Variables}
\end{table}
