
Communications between the Arduino and computer are done over the supplied RF link. The frequency is set to the group frequency specified. 
Further, the optional encryption was enabled with a randomly generated key, and the PAN ID was set to `6810'.

The communication follows the protocol described below. Communicated instructions are mapped to low level command, which are then executed on the Arduino. This is detailed in the following sections.

\subsection{Communication Protocol}

The communications protocol is formed of packets of the basic form: \\
\verb$<target><source><data><checksum>\r\n$.

\texttt{<target>} and \texttt{<source>} are the ASCII character `1' or `2' for the robots of group 11 and 12 accordingly, and the ASCII character `c' for the computer. For robot to computer messages, the targets `d' and `e' are also valid, used to send debug and error messages respectively.
\texttt{<data>} is the base64 encoded binary message, and \texttt{<checksum>} is a checksum of said message. The base64 encoding does not use padding bytes, but is otherwise standard.

The checksum consists of 2 characters, which are calculated as the checksum of all characters at even (starting 0) and odd indicies respectively, including \texttt{<target>} and \texttt{<source>}.
These checksums are calculated by retrieving the value of each the character in base64 encoding in the integer range 0-63, and XORing these together.
The resulting value from 0-63 is then encoded back into a base64 character.
For example, the checksum of `\texttt{dt6bas2}' is `\texttt{La}'.

All devices ignore all packets not addressed to it, as well as malformed packets. The validity of a packet is verified by the packet structure and the checksum.
Upon successful receipt of a packet, the recipient sends an acknowledgement to the sender.
An acknowledgement packet has the form `\verb%<target>$<checksum>\r\n%'. 
Acknowledgement packets are not acknowledged again.

The computer side will resend packets which have not been acknowledged.
The Arduino side does not verify packet receipt or resend packets, and discards packets which are longer than the internal buffer (60 bytes). 
To prevent flooding, only the last sent packet is resent. This suits the application well, as previously sent packets contain instructions to the robot which are superseded
by the more recent one.

