\documentclass[11pt,a4paper]{article}
\usepackage[utf8]{inputenc}
\usepackage{amsmath}
\usepackage{amsfonts}
\usepackage{amssymb}
\usepackage{graphicx}
\usepackage[left=2cm,right=2cm,top=2cm,bottom=2cm]{geometry}
\author{SDP Group 12}
\title{Technical Specification}

\begin{document}
\maketitle
Group 12 Technical Specification

\maketitle 
Hardware

\maketitle
Wheelbase
The robot uses a 3 wheeled design with 2 parallel 'drive' wheels along one 
axis of rotation (rotating independently) and a third 'turn' wheel rotating 
perpendicular to these 2 (see images). The 'origin' of the robot should be 
directly between the 2 drive wheels which are 15cm apart. The 'turn' wheel 
is 11cm back from the origin, the left wheel 7.5cm to the left and the right
7.5cm to the right. 

\maketitle
Front Assembly

The robot Has a 4 motor assembly mounted on 4 vertical bars coming up from the
struts joining the left and right NXT motors. These are arranged in 2 sets of 2.
The first set powering the kicker and the 2nd powering the grabbers.

\maketitle 
Grabbers

The grabbers consist of 2 long arms at different heights so they will not collide.
They have an angle at the end (see pictures) and the left is a little longer so
it does not strike the axis of rotation of the right.

\maketitle
Kicker

The kicker is powered by 2 motors on the front assembly with power supplies joined
(in inverted polarity) and powered by the same channel on the motor driver board to
keep them synchronised. The 2 motors are connected to each other and a large gears
(?? teeth) meshing with the small gear on the axis of rotation of the kicker. This
increases the speed of kick for the motors.

\maketitle 
Board and Battery Assembly

The boards are mounted on a platform between the 3 wheels. The arduino and power
shield are mounted in the centre with 2 battery packs of 3 mounted sideways on either
side and a 2 mounted between the 2 drive wheels. The rotary encoder board is
mounted on the back of the right battery pack and the motor driver board is mounted
on the left battery pack. As the batteries face in the packs must be detatched to
replace them. Both packs clip onto the platform with lego connectors.

\maketitle 
Motors

The 2 drive motors are lego NXT motors and have distance sensors built in, the
turn motor does not have a sensor built in and one has been mounted on the axle. 
\end{document}
