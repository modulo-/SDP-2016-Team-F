\documentclass[12pt,a4paper,titlepage]{article}
\usepackage[utf8]{inputenc}
\usepackage{amsmath}
\usepackage{amsfonts}
\usepackage{amssymb}
\usepackage{graphicx}
\usepackage{tabularx}
\usepackage[left=2cm,right=2cm,top=2cm,bottom=2cm]{geometry}
\usepackage{titlesec}
\usepackage{listings}
\lstset{language=Python, showstringspaces=false}


\usepackage[hypcap]{caption}
\usepackage[pdfusetitle, hidelinks]{hyperref}
\hypersetup{ 
    bookmarksnumbered=true,     
    bookmarksopen=true,         
    bookmarksopenlevel=2,         
    pdfstartview=Fit,           
    pdfpagemode=UseOutlines,    % this is the option you were lookin for
    pdfpagelayout=SinglePage
    }
    
\usepackage{float}

\author{SDP Group 12}
\title{Technical Specification}

\begin{document}
\maketitle

%Group final report No more than 12 pages, worth 16 (out of 50) marks

%Reports have three purposes: to provide sufficient information to assess your work; to document for the group and yourself your design and what you have done; and to provide useful information for anyone who wants to build on what you have done in future. Reports may contain detailed appendices beyond the page limit, particularly in aid of the last two goals; but markers are not required to read these appendices. A common mistake is to overuse the appendices— figures, tables, or other material that explains your work belongs in the main report, not an appendix. Large illustrations will not count towards the page limits.


\section{Introduction}

The aim of this project was to create a robot that would play a simplified version of 2-a-side football. We were recommended to use the provided Arduino Uno (Xino RF) board and LEGO pieces, a vision feed of the testing area (pitch), and computers to connect the two and implement any further logic.

This report will examine the decisions made in the design of each part and their impacts. Furthermore it will explain the principles our work relies on.


\subsection{Work Distribution and Management}
Overall the group was efficiently split into subgroups working on specific tasks. As opposed to many other groups, all team members have made significant contributions to the outcome of the project.

Furthermore, many tasks were shared with Group 11, and large contributions introduced by teamwork between the two halves of the team.


%\subsection{Planning}
%Mostly worked on planning.
%
%\subsection{Joel Hutton}
%Responsible for hardware.
%
%\subsection{Hugh McGrade}
%Leader of planning development.
%
%\subsection{Margus Lind}
%All around
%
%\subsection{Sanchit Gupta}
%Mostly worked on vision.
%
%\subsection{Simeon "King of Calibration" Ivanov}
%Leader of vision development.

\section{Vision}
\subsection{Requirements}
What and how to install

\subsection{Calibration}
By default the vision platform will launch a calibration routine for getting the most correct colour thresholds. Click as requested.

\subsection{Usage}
How to launch the vision portion and how to interface with it to get the world representation
Planner callback, getworldmodel method

\documentclass{article}
\usepackage{listings}
\begin{document}
\lstset{language=Python, showstringspaces=false}

\section{Planning}

\subsection{Design}

The planner uses a system of goals and actions. Goals define an overall strategy in a given situation leading to a particular aim. Actions define one instruction given to the robot with preconditions for its execution. Goals are composed of a sequence of actions. A goal object selects the next action required to achieve its aim by traversing its ordered list of actions and evaluating their preconditions.

In the implementation, all goals and actions are derived from their respective super classes. These are defined as follows:

\begin{lstlisting}
class Goal(object):
    '''
    Base class for goals
    '''
    def __init__(self, world, robot):
        self.world = world
        self.robot = robot

    # Return the next action necesary to achieve the goal
    def generate_action(self):
        info("Generating action for goal: {0}".format(self.__class__.__name__))
        for a in self.actions:
            if a.is_possible():
                return a
        return None


class Action(object):
    '''
    Base class for actions
    '''
    preconditions = []

    def __init__(self, world, robot, additional_preconds=[]):
        self.world = world
        self.robot = robot
        self.preconditions = self.__class__.preconditions + additional_preconds

    # Test the action's preconditions
    def is_possible(self):
        info("Testing action : {0}".format(self.__class__.__name__))
        for (condition, name) in self.preconditions:
            if not condition(self.world, self.robot):
                info("Precondition is false: {0}".format(name))
                return False
        info("Action possible: {0}".format(self.__class__.__name__))
        return True

    # Do comms to perform action
    def perform(self, comms):
        raise NotImplementedError

    # Get messages relating to action
    def get_messages(self):
        return []

    def get_delay(self):
        return DEFAULT_DELAY

\end{lstlisting}

The planner makes its decision based on only on the world state passed to it. This is described by an instance of the \lstinline|World| class passed to the planner. This object's state is updated by passing a new set of positions (as a dictionary of vectors) to the \lstinline|update_positions| method of \lstinline|World|. The \lstinline|World| class describes the state of the world from the vision, including position vectors for the robots, the ball and the goals. The \lstinline|World| class and its associated classes also provide methods on their data providing the planner with information about the world. Further utility functionality can be found in ``utils.py''

The overall planner is used by calling \lstinline{plan_and_act(world)}. This selects a goal based on the given world state  using its \lstinline|get_goal()| method. From this goal an action is generated using \lstinline|generate_action()|. The method then runs (using \lstinline|actuate(action)|) and returns a delay giving the time until the planner should be run again.

\subsection{Implementation}

Our attacker robot's goals and their respective actions are as follows:

AttackPosition

\begin{tabular}{ | l | l | }
\hline
Action & Preconditions \\ \hline
TurnToDefenderToReceive & Attacker in score zone \\ \hline
GoToScoreZone & Attacker is facing score zone  \\ \hline
TurnToScoreZone & None \\
\hline
\end{tabular}

Score

\begin{tabular}{ | l | l | }
\hline
Action & Preconditions \\ \hline
Shoot & Attacker has ball and attacker can score \\ \hline
TurnToGoal & Attacker has ball \\
\hline
\end{tabular}

GetBall

\begin{tabular}{ | l | l | }
\hline
Action & Preconditions \\ \hline
GrabBall & Attacker can catch ball and attacker's grabbers are open \\ \hline
GoToGrabStaticBall & Ball is static, attacker is facing ball and attacker's grabbers are open \\ \hline
OpenGrabbers & Ball is in attacker's grab range and attacker's grabbers are closed \\ \hline
GoToBallOpeningDistance & Attacker is facing ball \\ \hline
TurnToBall & None \\
\hline
\end{tabular}

AttackerBlock

\begin{tabular}{ | l | l | }
\hline
Action & Preconditions \\ \hline
TurnToBlockingAngle & Attacker in blocking position \\ \hline
GoToBlockingPosition & Attacker is facing blocking position \\ \hline
TurnToFaceBlockingPosition & None \\
\hline
\end{tabular}

The planner chooses a goal based for certain situations, as described in following table:

\begin{tabular}{ | l | l | }
\hline
Robot in possession & Goal \\ \hline
Our attacker & Score \\
Our defender & AttackPosition \\
Their attacker & AttackPosition \\
Their defender & AttackerBlock \\
Ball free & GetBall \\
\hline
\end{tabular}


\subsection{Extensability}

Extending the planner is simply a case of adding goals and actions then adding the logic to select these in \lstinline{select_goal(world)}. Any new goals or actions should subclass Goal and Action respectively and override methods were stated.

In actions, the logic for performing should be placed in \lstinline{perform(comms)}. This method is passed a \lstinline{CommsManager} object through which the robot can be sent instructions. No more than one call should be made to this object in any given action. A new action should also override \lstinline{get_delay} giving an appropriate delay (in seconds) before the planner should run again. New actions can also have preconditions defined in a variable \lstinline{preconditions}.

In goals, it may only be neccessary to write a Goal subclass with an ordered list of actions (from last to first). Otherwise the \lstinline{generate_action()} method can be overriden but similar logic should be followed.

\subsection{Integration}

An instance of the \lstinline{Planner} is kept by the ``main.py'' script. Based on the delays given by the planner, it calls the planner at varying intervals, passing it the latest world model provided by the vision. The planner uses a \lstinline{CommsManager} object to make control the robot.

\end{document}


\section{Communications}
How different parts talk to one another

\section{Hardware}

\begin{figure}[H]
\begin{subfigure}{.5\textwidth}
\centering
\includegraphics[width=\textwidth]{DSC_0036.jpg}
\caption{Wheelbase}
\label{pic:wheelbase}
\end{subfigure}
\begin{subfigure}{.5\textwidth}
\centering
\includegraphics[width=\textwidth]{DSC_0031.jpg}
\caption{Front Assembly}
\label{pic:front}
\end{subfigure}
\caption{Robot Construction}
\label{pic:construction}
\end{figure}

\subsection{Wheelbase}
\subsubsection{Dimensions}
'Drive' wheelbase:		15cm \newline
Radius of 'turn' wheel: 11cm
\subsubsection{Origin}
The origin is directly between the 'Drive' wheels
\subsubsection{Arrangement}
The robot uses a 3 wheeled 'T' design with 2 parallel 'drive' wheels along one axis of rotation (capable of rotating independently) and a third unpowered 'turn' wheel rotating perpendicular to these 2 (\autoref{pic:wheelbase}). The 'origin' of the robot should be 
directly between the 2 drive wheels which are 15cm apart. The 'turn' wheel 
is 11cm back from the origin, the left wheel 7.5cm to the left and the right
7.5cm to the right. 

\subsubsection{Motors}
The 2 drive motors are lego NXT motors and have rotational encoders built in, this is used to turn and move accurately. 



\subsection{Front Assembly}
\subsubsection{Description}
The robot Has a 2 motor assembly mounted on 4 vertical bars coming up from the struts joining the left and right NXT drive motors. These are symmetrically mounted and linked by a chain in order to keep them synchronised. The right grabber is lower than the left one so that they do not hit into each other. On one of the centre axles there is a rotary encoder to give the position of the grabbers to the Arduino, through the rotary encoder board. Position 0 is open, 10 is closed with ball and 13 is closed without ball. See \autoref{pic:front}.

The grabber assembly is fully detachable and can be clipped on and off. 


\subsection{Kicker}
\subsubsection{Position}
The kicker is a solenoid mounted on the underside of the robot (see \autoref{pic:construction}) connected through a relay to the battery packs, with the relay signal going to the power shield and being controlled directly by the Arduino. on the end of the solenoid is a curved kicker designed to keep the ball straight.

\subsubsection{Power}
The kicker is capable of kicking a distance of 3.2m at full power with an approx error of 15 percent

\subsection{Board and Battery Assembly}
\subsubsection{Structure}
The boards are mounted on a platform between the 3 wheels. The Arduino and power
shield are mounted in the centre with 2 battery packs of 2 18650 Li-Ion batteries mounted sideways on either
side. The rotary encoder board is
mounted on the back of the right battery pack and the motor driver board is mounted
on the left battery pack. As the batteries face in the packs must be detached to
replace them. Both packs clip onto the platform with lego connectors.

\subsubsection{Batteries}
The batteries are 4* 18650 Canwelum Li-Ion cells wired in series in 2 battery packs. They can be removed and charged individually in the Li-Ion charger.

\begin{tabularx}{\textwidth}{XX}
\textbf{Voltage (per cell):}	& 2.65 - 4.2 \\
\textbf{Voltage (total):}		& 10.6 - 16.8 \\
\textbf{Max Current:}			& 7A \\
\textbf{Internal Resistance (per cell)}: & 0.150 Ohms \\
\textbf{Internal Resistance (total)}: & 0.6 Ohms \\
\textbf{Capacity}: & 2250mAh \\
\textbf{Protection}: & overvolt and undervolt \\
\end{tabularx}
    
\subsubsection{Connections}
The batteries are connected in series and must be connected to the power board through the yellow switch, with the splitters on the line powering the solenoid relay (orange wires)
\subsubsection{Switch}
A yellow switch turns the robot on and off

\end{document}